\section{Appendix A - Operating System Processes Configuration Script}
\label{sec:appendA}
\begin{verbatim}
#!/bin/sh
#systemctl list-unit-files --type=service --state=enabled

#chkconfig service (off, stop, start, restart)
#systemctl (stop, start, restart, status) service

#off services, stop service
systemctl stop abrt-ccpp.service
systemctl stop abrt-oops.service
systemctl stop abrt-vmcore.service
systemctl stop abrt-xorg.service
chkconfig atd off
systemctl stop atd.service
#accounts-daemon.service???
systemctl stop avahi-daemon.service
chkconfig avahi-daemon off
systemctl stop bluetooth.service
chkconfig bluetooth off
systemctl stop crond.service
chkconfig crond off
systemctl stop cups.service
chkconfig cups off
systemctl stop dbus-org.bluez.service
chkconfig dbus-org.bluez off
systemctl stop dbus-org.freedesktop.Avahi.service
systemctl stop dbus-org.freedesktop.ModemManager1.service
systemctl stop dbus-org.freedesktop.nm-dispatcher.service
systemctl stop dmraid-activation.service
chkconfig dmraid-activation off
systemctl stop getty@.service
chkconfig getty@ off
systemctl stop libvirtd.service
chkconfig libvirtd off
systemctl stop lvm2-monitor.service
systemctl stop mcelog.service
systemctl stop mdmonitor.service
systemctl stop ModemManager.service
chkconfig ModemManager off
systemctl stop multipathd.service
chkconfig multipathd off
systemctl stop phc2sys.service
systemctl disable phc2sys.service
systemctl stop vmtoolsd.service
chkconfig vmtoolsd off
systemctl stop vsftpd.service
chkconfig vsftpd off

#kill processes
killall abrt-applet
killall at-spi2-registryd
killall at-spi-bus-launcher
killall caribou
killall evolution-alarm-notify
killall gconfd-2
killall gnome-keyring-daemon
killall gnome-software
killall gsd-printer
killall gvfs-afc-volume-monitor
killall gvfsd-metadata
killall gvfsd-trash
killall gvfs-goa-volume-monitor
killall gvfs-gphoto2-volume-monitor
killall gvfs-mtp-volume-monitor
killall gvfs-udisks2-volume-monitor
killall gvfs-mtp-volume-monitor
killall gvfs-udisks2-volume-monitor
killall obexd
killall pulseaudio
killall seapplet
killall systemd
killall tracker-extract
killall tracker-miner-apps
killall tracker-miner-fs
killall tracker-miner-user-guides
killall tracker-store

\end{verbatim}

\clearpage
\section{Appendix B - RabbitMQ Configuration}
\label{sec:appendB}
To configure RabbitMQ you must follow these steps:
\begin{itemize}
	\item Install RabbitMQ-Server.
	\item Enable the RabbitMQ manage plugin through command \textit{"rabbitmq-plugins enable rabbitmq\_management"}.
	\item Create probes configured in source code and the queues required by PaSCAni.
	\item Ensure accessibility from all processing nodes. To accomplish this step, you must copy file \textit{"rabbitmq.config"} in path \textit{"/etc/rabbitmq/"}.
	\item Publish 5672 TCP port used by RabbitMQ in all processing nodes.
\end{itemize}

\clearpage
\section{Appendix C - Java Virtual Machine (JVM) Configuration}
\label{sec:appendC}
The Java Virtual Machine memory is configured to 6 GB through the next command \\

\textbf{\textit{"export \_JAVA\_OPTIONS="-Xms6g -Xmx6g" "}}.
